\documentclass[twocolumn,twoside]{IEEEtran}

\usepackage{graphicx}
\usepackage{svg}


\usepackage[framemethod=TikZ]{mdframed}% Required for creating boxes
\usepackage{amsmath}

\mdfdefinestyle{left}{
    linecolor=black, % Outer line color
    outerlinewidth=0.1pt, % Outer line width
    roundcorner=0pt, % Amount of corner rounding
    innertopmargin=10pt, % Top margin
    innerbottommargin=10pt, % Bottom margin
    innerrightmargin=10pt, % Right margin
    innerleftmargin=10pt, % Left margin
    backgroundcolor=white, % Box background color
    frametitlebackgroundcolor=white, % Title background color
    frametitlerule=false, % Title rule - true or false
    frametitlerulecolor=white, % Title rule color
    frametitlerulewidth=0.5pt, % Title rule width
    frametitlefont=\Large, % Title heading font specification
    font=\small,
    rightline=false,
    leftline=true,
    topline=false,
    bottomline=false
}

\newread\myinput

\newcommand{\showbayes}[1]{
% We use '\jobname.temp' to create a uniquely-named temporary file
\immediate\write18{cat ../Bayes/#1/net.txt.k > '\jobname.temp'}
\openin\myinput=\jobname.temp
% The group localizes the change to \endlinechar
\bgroup
  \endlinechar=-1
  \read\myinput to \localline
  % Since everything in the group is local, we have to explicitly make the
  % assignment global
  \global\let\kbayes\localline
\egroup
\closein\myinput
% Clean up after ourselves
\immediate\write18{rm -f -- '\jobname.temp'}

\center
Bayes
\includegraphics[width=\linewidth]{../Bayes/#1/#1.png}

$K=\kbayes$

}

\newcommand{\showmdl}[1] {
% We use '\jobname.temp' to create a uniquely-named temporary file
\immediate\write18{../get.py ../MDL/#1/mat.p > '\jobname.temp'}
\openin\myinput=\jobname.temp
% The group localizes the change to \endlinechar
\bgroup
  \endlinechar=-1
  \read\myinput to \localline
  % Since everything in the group is local, we have to explicitly make the
  % assignment global
  \global\let\kmdl\localline
\egroup
\closein\myinput
% Clean up after ourselves
\immediate\write18{rm -f -- '\jobname.temp'}

\center
MDL
\includegraphics[width=\linewidth]{../MDL/#1/#1.png}

$K=\kmdl$

}

\newcommand{\showgraph}[1]{
\vspace{10pt}
\begin{minipage}{0.45\linewidth}

\showmdl{#1}

\end{minipage}
\begin{minipage}{0.45\linewidth}

\showbayes{#1}

\end{minipage}
\vspace{10pt}

\def\currentgraph{#1}
}

\newcommand{\showlearning}[1]{
    \includegraphics[width=\linewidth]{../Bayes/#1/figure.png}
}

\newcommand{\showgraphwithparams}[2]{
    \vspace{3pt}

    \begin{tabular}{l | r}
        \hline \\
        \begin{minipage}{0.2\linewidth}

        #1

        \end{minipage} &
        \begin{minipage}{0.7\linewidth}
    
            \showgraph{#2}
    
        \end{minipage} \\
       \hline
    \end{tabular}

    \vspace{3pt}
}

\newcommand{\showgraphwithtitle}[2]{
    \begin{minipage}{\linewidth}
        \vspace{10pt}
        \center #1

        \showgraph{#2}
    \end{minipage}
}


% \usepackage{minipage}

\nonstopmode \title{Comparison of Variational Bayes and Minimum Description
Length Algorithms for Determining $k$ in Undirected Networks}

\author{Brett Israelsen \& Joshua Rahm}


\begin{document}
\maketitle

\begin{abstract}
Herin are compared two algorithms for discovering
the number of groups, $k$, in a network given no prior information about the
structure of the network. The two algorithms we have tested in this project are
the Minimum Description Length (MDL)\cite{Peixoto2013} and Variational Bayes (VB)\cite{Hofman2008}.
\end{abstract}

\section*{Introduction}\label{sec:Intro} In network analysis it is frequetntly
desirable to identify communities (often referred to as blocks, modules, or
clusters \footnote{Cluster detection is not the same as community detection but
is an analog in the domain where data are not connected to each other in the style
of networks.}). Community detection involves finding groups of vertices that
are related in some way. This can be thought of as identifying similar characteristics,
these properties may not, be and in fact are typically not formally defined. In
other words they are inferred by the detection algorithm.

Modeling methods like the stochastic block-model (SBM) require an estimate of 
the number of blocks or communities. Without the use of a community detection
algorithm this task falls on the individual doing the analysis to intuit or
guess the number of communities in the network. Typically this is not a simple
task especially in the case of large real-world networks.

% prehaps put a ridiculogram here to show what a crazy real-wrold network looks like

It is useful then to have an algorithm that can analyze the network in an
algorithmic manner and identify the number of communities in a graph so that
models like SBM can then be used to model the network and generate other
networks for analysis.

% \emph{original}
% Sometimes, when given a network, it is useful to find community structure in a
% network in order to be able to do further analysis into the structure of the
% network. Many times, little to no prior information is given about the state of
% the network is known. Given this as the case, standard modularity measures like
% greedy agglomeration and the KL heuristic are more difficult to use because of their
% sensitivity to degree such as in the famous Karate Club network.
% 
% To combat this, there are several algorithms used to estimate the number of groups ($k$)
% in a network, an from there, a reduction algorithm may be used to determine the association
% of each vertex in each of the $k$ groups.
% 
% In this paper, we have compared the results obtained from two such algorithms;
% Variational Bayes and Minimum Description Length.
% 
\section*{Background}\label{sec:Background} Several methods have been proposed
to do community detection. One method that is well known is Modularity
maximization \cite{Newman2004}. However, through much analysis and application
this method has been known to have some drawbacks such as not being able to
recover bipartite structure \cite{Peixoto2013} or recover communities based on
degree structure. It is also limited by resolution 
\cite{fortunato2007resolution}.

Other methods are based around using generative models (such as the SBM), and
then obtain the most likely block model given some criteria. Several of these
methods have also been shown to have resolution
limits\cite{fortunato2007resolution} that inhibit their ability to detect
communities under a certain size. This severely inhibits their use as a general
tool in network analysis and modeling. Another problem with some methods is
that they require other parameters to be known a priori.

Two methods that currently stand out as performing well are the Minimum
Description Length (MDL) \cite{Peixoto2013} and variational Bayes (VB)
\cite{Hofman2008}. Neither of these methods exhibits the resolution limit
problems or require other network information to be known before analysis.

\emph{MDL - } MDL was first introduced with application to SBM in
\cite{Rosvall2007}. The key concept of MDL is that the best model for a network
is the model that most compresses it. Thus MDL find the number of communities
that will allow the network to be represented with the least amount of
information. In \cite{Peixoto2013} a general algorithm was introduced to
execute this algorithm on a general network.

\emph{VB - } VB was introduced by Hofman and Wiggins in \cite{Hofman2008}. This
method was shown to be the general version several other existing methods.
This method is based on Bayesian methods for model selection. Using Bayesian
techniques then the basic idea of VB is to infer a number of
communities(referred to as modules in the paper) that is most probable and then
infer posterior distributions over model parameters. In the paper VB is shown
to perform better that Modularity specifically with regards to the resolution
limit.

While each of these two methods has been shown to behave well in some
situations there has been little or no work (according to our knowledge and
investigation) that investigates how the differences between these two
approaches may affect their use/desirability in varied real-world networks.

Our aim in this paper is to document our investigation into the similarities
and differences between these two approaches for detecting the number of
communities in various networks. Our hope is that this information will help practitioners to
choose between the two methods when applying them to data sets.

\section*{Approach}\label{sec:Approach} Stated again, our aim is to compare the
performance of the MDL and VB algorithms on several simulated networks and some
real networks.

We were able to make extensive use of code produced by the authors of both MDL
and VB \cite{peixoto_graph-tool_2014} and \cite{Hofman_python_2008}. Our first
goal was to re-produce some of the key results from both the papers
\cite{Peixoto2013} and \cite{Hofman2008}, and then as the examples were
different in both papers we compared them against each other.

From there we moved to a deeper comparison between the two methods. Several of
the canonical graph models were used in the attempt to pinpoint or decouple
the effects of specific graph properties on the performance of each of the
methods. The canonical graph types we used are: binary trees, Prices networks
(preferential attachment), random graphs (Erd\H{o}s-R\'{e}nyi with fixed degree
vertices, and Poisson distributed degree), block model generated graphs, and a
class of clique-based graphs that are designed as resolution limit tests.

The key result of our work should be considered the value $K$(number of
communities) given by each algorithm on the different classes of networks. However, in an attempt to more
fundamentally understand the effects of the choices of $K$ we used an SBM to
fit the respective graphs and then analyze the properties of the resulting
communities.

Our results and discussion are presented as follows: first we address the
general behaviors of the two algorithms on simulated networks. From there we
discuss the results on some interesting real networks.

\section*{Results \& Discussion}\label{sec:RD}



% For tree graphs:
%
% n_i over N for each group i as a histogram
% Distribution of level by community
% 
% For Ring Graphs:
%
%   n_i over n_c
%   m_i over (choose n_c 2)
%   average degree for MDL
%
% Price:
%   
% Poisson:
%   Average degree of each group

\newcommand{\inputsamepage}[1] {
    \begin{minipage}{\linewidth}
        \input{#1}
    \end{minipage}
}


\break
\section*{Binary Trees}
Our first set of synthesized data is on different sets of
binary trees of different depths $l$. We have taken data from
the all positive $l \le 10$. These graphs are all regular binary
trees with no entropy or randomness.

After building these graphs, it became apparent that both MDL and
VB were detecting the bipartite nature of the trees.

\begin{minipage}{\linewidth}

$l=5$

\showgraph{Tree_c_2_l_5}

$l=7$

\showgraph{Tree_c_2_l_7}

$l=10$

\showgraph{Tree_c_2_l_10}
\end{minipage}

These are an interesting set of graphs. They are all binary trees with
the levels ($l$) 5, 7 and 10 respectively.

with MDL, the results are simple. The graph of level 5 is found to have
no groups, but after that, MDL finds the disassortative pattern which
is the bipartite nature of trees; the odd generations are a part of
group $A$ and the even generations are a part of a second group $B$.

VB contains the more interesting results now. Noticably, VB has run away
on the tree of depth 5, reporting each node as its own group.  As it
turns out, it does this for depths of 4 ond 6 as well. Beyond that, the
algorithm follows an interesting pattern where it, like MDL, finds a
disaccociative pattern, but unlike MDL it has multiple \emph{pairs} of
groups that alternate with eachother. These pairs exist independently in
separate branches of the tree. This causes the number of groups in the
graph to grow roughly linearly with the number of vertices in the graph

To illustrate this, the graph below shows how the number of groups $k$ is
affected by the depth of the tree:

\includesvg[width=\linewidth]{kvslev}

Regarding the value of $l$ that caused VB to diverge. Examining the
learning curve yeilds an insightful result.

\showlearning{Tree_c_2_l_5}

It is clear that the number of groups should have been chosen to be 5,
however, due to the non-greedy nature of the VB algorithm, it errored
on the side of a larger number of simpler groups. To patch this issue,
we introduce a modification to VB to make it greedy. Using this new greedy
approach, we can see the results of re computing the groups for $k=5$

\begin{minipage}{\linewidth}
\showbayes{Tree_c_2_l_5_greedy}
\vspace{10pt}
\end{minipage}

The greedy approach allows VB to value fewer groups as being simpler than
having many more groups. Using this new, greedy approach, we may re-model
the results to rebuild the plot from above.

\includesvg[width=\linewidth]{kvslev_greedy}

The exponential relationship is easier to see now that the graph
has been fixed.


\break
\section*{Ring Graphs (Resolution Limit Test)}
Ring graphs are graphs which consist of cliques of nodes
around a ring. These graphs are trivial to find groups in
just by looking at them.


\break
\section*{Erd\H{o}s-R\'{e}nyi Random Graphs}
We create the set of random graphs using the two parameters
of probability $p$ and number of nodes $n$

Examples of such graphs are as follows

\showgraphwithparams{
{\begin{align*}
n &: 20 \\
p &: 0.2
\end{align*}}
}{Random_20_20}

\showgraphwithparams{
{\begin{align*}
n &: 100 \\
p &: 0.1
\end{align*}}
}{Random_100_10}

\showgraphwithparams{
{\begin{align*}
n &: 100 \\
p &: 0.22
\end{align*}}
}{Random_100_22}

When we fix the number of nodes to $50$ and set $p$ to be our
free variable, we can get a sense of how these algorithms find
groups.

\includesvg[width=\linewidth]{random}

For all the tested values, MDL only found 1 group. This is understandable
as the behavior of MDL when there is no discernible structure is to
return only a single group. In this sense, MDL was perfect in calling
the bluff. VB is not as such. As the diagram shows, as the random
graph approaches a clique, VB approaches to $k=1$. However, the road to
that point is very variable and almost random itself.


\section*{Block Model Generated Graphs}
For the final synthetic data set, we will generate new graphs from block
model graphs. Each block model graph will come with four parameters:
number of vertices $n$, the number of groups $g$, probability of
connecting internal to a block $p_{in}$ and probability of connecting to
another block $p_{out}$.

Some examples of such graphs are as follows:

\showgraphwithparams{

{\begin{align*}
n &: 100 \\
g &: 3 \\
p_{in} &: 0.2 \\
p_{out} &: 0.001
\end{align*}}

}{BM_n_100_g_3_pi_20_po_0}

\showgraphwithparams{
{\begin{align*}
n &: 100 \\
g &: 3 \\
p_{in} &: 0.01 \\
p_{out} &: 0.10
\end{align*}}
}{BM_n_100_g_3_pi_1_po_10}

\showgraphwithparams{
{\begin{align*}
n &: 350 \\
g &: 7 \\
p_{in} &: 0.05 \\
p_{out} &: 0.001
\end{align*}}
}{BM_n_350_g_7_pi_5_po_0}

For the first set of tests, we set the parameters $n=200$, $g=4$, and
$p_{out} = 1-p_{in}$. As such, our free parameter is $p_{in}$. Varying this

\includesvg{k_vs_p_out}

This graphs gives a sense of the resolution limit of the different
algorithms. For graphs of a more assortative nature ($p_{in} > p_{out}$),
Bayes has a higher resolution than MDL, and is able to discover groups
even as they become almost indistinguishable. This changes as the network
becomes much more disassortative ($p_{out} > p_{in}$). MDL is the first
to detect the emergence of multiple groups. Bayes on the other hand never
fully emerges.

It is also important to note that MDL is very consistent. The only two
value of $k$ which are represented are 4 and 1. VB, on the other hand,
detects a range of groups having 1, 2, 4 and even 5 represented.




\section*{Well Known Graphs}
Finally, no discussion would be complete without a discussion of
the applications and interpretations of these results on real networks.

For example, one of the most interesting networks seen for both algorithms
is the Serengeti Food Web\cite{Baskerville2011}.

\showgraphwithtitle{Serengeti Food Web}{serengeti-foodweb}

In this network, MDL found 4 solid groups. It seems to have detected
two ``metagroups'', one for each side. Each metagroups is then broken up
into a core-periphery structure with two subgroups. It seems that MDL
has found two ecosystems with their own set of predators and prey.

In this example, VB is not so lucky. It has not detected any structure. This
is very apparent from the complexity curve.

\showlearning{serengeti-foodweb}

\showgraphwithtitle{Zachary Karate Club}{karate}

The famous karate club network\cite{Zachary1977} is almost the opposite of the Serengeti
Foodweb. In this network, MDL does not find enough structure to merit the
creation of more than one group. However, VB found enough structure to
find five groups. Like MDL in the food web, VB has found 3 top level groups
of which two are broken into a core-periphery structure.

The dolphin data set\cite{Lusseau2003}

This graph is an interesting one because MDL has managed to find the 'correct'
partition. This is great, but in fact VB found a more interesting set of
things. The MDL algorithm seems to have found, like in the previous, certain
``metagroups'' that are sub-groups of the groups found by the MDL algorithm.

\showgraphwithtitle{Dolphins}{dolphins}

Without more knowledge of data within the network. It is hard for us to tell
what VB actually found. Perhaps it found the same two groups that MDL found
along with data about the social classes of the dolphins within those two
groups. Whatever the reason, Variational Bayes finds many more groups than MDL,
while retaining the ability to group these groups to create the same groups
that MDL found.

\section*{Conclusions}

There is no clear winner between MDL and VB. Both algorithms do a reasonable
job on some networks, while not so reasonable on others. The algorithm of
choice is dependent on the desired behavior. We have found that in general MDL
behaves in a more predictable way than VB. This may be exemplified in many of
the previous examples by the variation seen in the VB results. For example, for
some graphs with little to no structure will either cause VB to diverge into
assigning as many groups as there are vertices, or only one group is detected.
MDL on the other hands will always discover just one group for graphs with
inherently little structure.

VB excels at finding many groups, and when $k$ converges, many high resolution
structures may be picked out, generating much more information than MDL can. In
the Karate Club and Dolphins networks while MDL only picked out two or three
groups, VB was able to, in addition to finding those groups, recurse into those
groups and find subgroups, making it much more flexible as a whole.

With the flexibility VB offers, there a very real possibility of over fitting
the data, which is what was seen over and over again in the previous networks.
With this, MDL is much better at picking out the macro structure of the network
reducing the sway of random noise.

Over all, neither should always be used above the other. Rather, the modeler
should compare the pros and cons of each algorithm and pick the best for the
model and the data being obtained from the graph and what the potential risks
are





\onecolumn
\section{Appendix}
 \includesvg[width=\linewidth]{n_verts_vs_tree_level}

\twocolumn
\bibliography{CS5352_Project}{}
\bibliographystyle{plain}
\end{document}

Ring graphs are graphs which consist of cliques of nodes around a ring. To
construct the ring graphs, there are two parameters we have defined: the clique
size $c$ and the number of cliques in the ring $g$

Some examples of such graphs are as follows.

\begin{minipage}{\linewidth}

\showgraphwithparams{
{\begin{align*}
c &: 4 \\
g &: 2
\end{align*}}
}{RLT_c_4_g_2}

\showgraphwithparams{
{\begin{align*}
c &: 4 \\
g &: 10
\end{align*}}
}{RLT_c_4_g_10}

\showgraphwithparams{
{\begin{align*}
c &: 40 \\
g &: 30
\end{align*}}
}{RLT_c_40_g_30}

\end{minipage}

Sometimes, with these graphs, Bayes and MDL under estimate the number of
groups, however, often, Bayes does a good job detecting an expected group
structure. In order to further understand how each algorithm detects
these groups and when they find an unexpected answer, we calculated the
following plot which shows the number of groups detected as a function of
the number of groups around the ring, with the clique size fixed to
$c=4$.

\begin{minipage}{\linewidth}
\includesvg[width=\linewidth]{k_vs_g_c_4}
\end{minipage}

MDL never does a great job discovering the number of groups, so for these
graphs, Bayes is generally going to be the better choice; however, this
is not without reservation as there is some critical point at $g=24$ that
causes VB to stop finding more groups. The mechanism for this is not fully
understood.

Increasing the number of nodes per clique changes significantly the
results.

\begin{minipage}{\linewidth}
\includesvg[width=\linewidth]{k_vs_g_c_10}
\end{minipage}

Here it is obvious that MDL does better, keeping up with VB for $g < 20$,
but falls far behind beginning at the critical point of $g=20$.


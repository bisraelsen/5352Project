We create the set of random graphs using the two parameters
of probability $p$ and number of nodes $n$

Examples of such graphs are as follows

\showgraphwithparams{
{\begin{align*}
n &: 20 \\
p &: 0.2
\end{align*}}
}{Random_20_20}

\showgraphwithparams{
{\begin{align*}
n &: 100 \\
p &: 0.1
\end{align*}}
}{Random_100_10}

\showgraphwithparams{
{\begin{align*}
n &: 100 \\
p &: 0.22
\end{align*}}
}{Random_100_22}

When we fix the number of nodes to $50$ and set $p$ to be our
free variable, we can get a sense of how these algorithms find
groups.

\includesvg[width=\linewidth]{random}

For all the tested values, MDL only found 1 group. This is understandable
as the behavior of MDL when there is no discernible structure is to
return only a single group. In this sense, MDL was perfect in calling
the bluff. VB is not as such. As the diagram shows, as the random
graph approaches a clique, VB approaches to $k=1$. However, the road to
that point is very variable and almost random itself.

Our first set of synthesized data is on different sets of
binary trees of different depths $l$. We have taken data from
the all positive $l \le 10$. These graphs are all regular binary
trees with no entropy or randomness.

After building these graphs, it became apparent that both MDL and
VB were detecting the bipartite nature of the trees.

\showgraphwithparams{
{\begin{align*}
l &: 5 \\
\end{align*}}
}{Tree_c_2_l_5}

\showgraphwithparams{
{\begin{align*}
l &: 7 \\
\end{align*}}
}{Tree_c_2_l_7}

\showgraphwithparams{
{\begin{align*}
l &: 10 \\
\end{align*}}
}{Tree_c_2_l_10}

These are an interesting set of graphs. They are all binary trees with
the levels ($l$) 5, 7 and 10 respectively.

with MDL, the results are simple. The graph of level 5 is found to have
no groups, but after that, MDL finds the disassortative pattern which
is the bipartite nature of trees; the odd generations are a part of
group $A$ and the even generations are a part of a second group $B$.

VB contains the more interesting results now. Noticably, VB has run away
on the tree of depth 5, reporting each node as its own group.  As it
turns out, it does this for depths of 4 ond 6 as well. Beyond that, the
algorithm follows an interesting pattern where it, like MDL, finds a
disaccociative pattern, but unlike MDL it has multiple \emph{pairs} of
groups that alternate with eachother. These pairs exist independently in
separate branches of the tree. This causes the number of groups in the
graph to grow roughly linearly with the number of vertices in the graph

To illustrate this, the graph below shows how the number of groups $k$ is
affected by the depth of the tree:

\includesvg[width=\linewidth]{kvslev}

Regarding the value of $l$ that caused VB to diverge. Examining the
learning curve yeilds an insightful result.

\showlearning{Tree_c_2_l_5}

It is clear that the number of groups should have been chosen to be 5,
however, due to the non-greedy nature of the VB algorithm, it errored
on the side of a larger number of simpler groups. To patch this issue,
we introduce a modification to VB to make it greedy. Using this new greedy
approach, we can see the results of re computing the groups for $k=5$

\begin{minipage}{\linewidth}
\showbayes{Tree_c_2_l_5_greedy}
\vspace{10pt}
\end{minipage}

The greedy approach allows VB to value fewer groups as being simpler than
having many more groups. Using this new, greedy approach, we may re-model
the results to rebuild the plot from above.

\includesvg[width=\linewidth]{kvslev_greedy}

The exponential relationship is easier to see now that the graph
has been fixed.

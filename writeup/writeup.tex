\documentclass[twocolumn,twoside]{IEEEtran}

\usepackage{graphicx}
\usepackage{svg}


\usepackage[framemethod=TikZ]{mdframed}% Required for creating boxes
\usepackage{amsmath}

\mdfdefinestyle{left}{
    linecolor=black, % Outer line color
    outerlinewidth=0.1pt, % Outer line width
    roundcorner=0pt, % Amount of corner rounding
    innertopmargin=10pt, % Top margin
    innerbottommargin=10pt, % Bottom margin
    innerrightmargin=10pt, % Right margin
    innerleftmargin=10pt, % Left margin
    backgroundcolor=white, % Box background color
    frametitlebackgroundcolor=white, % Title background color
    frametitlerule=false, % Title rule - true or false
    frametitlerulecolor=white, % Title rule color
    frametitlerulewidth=0.5pt, % Title rule width
    frametitlefont=\Large, % Title heading font specification
    font=\small,
    rightline=false,
    leftline=true,
    topline=false,
    bottomline=false
}

\newread\myinput

\newcommand{\showbayes}[1]{
% We use '\jobname.temp' to create a uniquely-named temporary file
\immediate\write18{cat ../Bayes/#1/net.txt.k > '\jobname.temp'}
\openin\myinput=\jobname.temp
% The group localizes the change to \endlinechar
\bgroup
  \endlinechar=-1
  \read\myinput to \localline
  % Since everything in the group is local, we have to explicitly make the
  % assignment global
  \global\let\kbayes\localline
\egroup
\closein\myinput
% Clean up after ourselves
\immediate\write18{rm -f -- '\jobname.temp'}

\center
Bayes
\includegraphics[width=\linewidth]{../Bayes/#1/#1.png}

$K=\kbayes$

}

\newcommand{\showmdl}[1] {
% We use '\jobname.temp' to create a uniquely-named temporary file
\immediate\write18{../get.py ../MDL/#1/mat.p > '\jobname.temp'}
\openin\myinput=\jobname.temp
% The group localizes the change to \endlinechar
\bgroup
  \endlinechar=-1
  \read\myinput to \localline
  % Since everything in the group is local, we have to explicitly make the
  % assignment global
  \global\let\kmdl\localline
\egroup
\closein\myinput
% Clean up after ourselves
\immediate\write18{rm -f -- '\jobname.temp'}

\center
MDL
\includegraphics[width=\linewidth]{../MDL/#1/#1.png}

$K=\kmdl$

}

\newcommand{\showgraph}[1]{
\vspace{10pt}
\begin{minipage}{0.45\linewidth}

\showmdl{#1}

\end{minipage}
\begin{minipage}{0.45\linewidth}

\showbayes{#1}

\end{minipage}
\vspace{10pt}

\def\currentgraph{#1}
}

\newcommand{\showlearning}[1]{
    \includegraphics[width=\linewidth]{../Bayes/#1/figure.png}
}

\newcommand{\showgraphwithparams}[2]{
    \vspace{3pt}

    \begin{tabular}{l | r}
        \hline \\
        \begin{minipage}{0.2\linewidth}

        #1

        \end{minipage} &
        \begin{minipage}{0.7\linewidth}
    
            \showgraph{#2}
    
        \end{minipage} \\
       \hline
    \end{tabular}

    \vspace{3pt}
}

\newcommand{\showgraphwithtitle}[2]{
    \begin{minipage}{\linewidth}
        \vspace{10pt}
        \center #1

        \showgraph{#2}
    \end{minipage}
}


% \usepackage{minipage}

\nonstopmode \title{Comparison of Variational Bayes and Minimum Description
Length Algorithms for Determining $k$ in Undirected Networks}

\author{Brett Israelsen & Joshua Rahm}


\begin{document}
\maketitle

\begin{abstract}
We have compared two different algorithms for discovering
the number of groups, $k$, in a network given no prior information about the
structure of the network. The two algorithms we have tested in this project are
the Minimum Description Length (MDL) and Variational Bayes (VB).
\end{abstract}

\section*{Introduction}

Sometimes, when given a network, it is useful to find community structure in a
network in order to be able to do further analysis into the structure of the
network. Many times, little to no prior information is given about the state of
the network is known. Given this as the case, standard modularity measures like
greedy agglomeration and the KL heuristic are more difficult to use because of their
sensitivity to degree such as in the famous Karate Club network.

To combat this, there are several algorithms used to estimate the number of groups ($k$)
in a network, an from there, a reduction algorithm may be used to determine the association
of each vertex in each of the $k$ groups.

In this paper, we have compared the results obtained from two such algorithms;
Variational Bayes and Minimum Description Length.

\section*{Background}


\section*{Approach}

\section*{Results \& Discussion}



% For tree graphs:
%
% n_i over N for each group i as a histogram
% Distribution of level by community
% 
% For Ring Graphs:
%
%   n_i over n_c
%   m_i over (choose n_c 2)
%   average degree for MDL
%
% Price:
%   
% Poisson:
%   Average degree of each group
\subsection*{\underline{Well-Known Graphs}}

\newcommand{\inputsamepage}[1] {
    \begin{minipage}{\linewidth}
        \input{#1}
    \end{minipage}
}

We have run these algorithms on graphs which are well known. And have analyzed the
results.

\subsection*{Adjective-Nouns}
\inputsamepage{include/adjnoun.tex}

\subsection*{Bow-tie}
\inputsamepage{include/bowtie.tex}

\subsection*{Dolphins}
\inputsamepage{include/dolphins.tex}

\subsection*{Les Mis\'{e}rables}
\inputsamepage{include/lesmis.tex}

\subsection*{Football}
\inputsamepage{include/football.tex}

\subsection*{Zachary Karate Club}
\inputsamepage{include/karate.tex}

\subsection*{Political Books}
\inputsamepage{include/polbooks.tex}

\subsection*{Serengeti Foodweb}
\inputsamepage{include/serengeti-foodweb.tex}

\break
\section*{Binary Trees}
Our first set of synthesized data is on different sets of
binary trees of different depths $l$. We have taken data from
the all positive $l \le 10$. These graphs are all regular binary
trees with no entropy or randomness.

After building these graphs, it became apparent that both MDL and
VB were detecting the bipartite nature of the trees.

\begin{minipage}{\linewidth}

$l=5$

\showgraph{Tree_c_2_l_5}

$l=7$

\showgraph{Tree_c_2_l_7}

$l=10$

\showgraph{Tree_c_2_l_10}
\end{minipage}

These are an interesting set of graphs. They are all binary trees with
the levels ($l$) 5, 7 and 10 respectively.

with MDL, the results are simple. The graph of level 5 is found to have
no groups, but after that, MDL finds the disassortative pattern which
is the bipartite nature of trees; the odd generations are a part of
group $A$ and the even generations are a part of a second group $B$.

VB contains the more interesting results now. Noticably, VB has run away
on the tree of depth 5, reporting each node as its own group.  As it
turns out, it does this for depths of 4 ond 6 as well. Beyond that, the
algorithm follows an interesting pattern where it, like MDL, finds a
disaccociative pattern, but unlike MDL it has multiple \emph{pairs} of
groups that alternate with eachother. These pairs exist independently in
separate branches of the tree. This causes the number of groups in the
graph to grow roughly linearly with the number of vertices in the graph

To illustrate this, the graph below shows how the number of groups $k$ is
affected by the depth of the tree:

\includesvg[width=\linewidth]{kvslev}

Regarding the value of $l$ that caused VB to diverge. Examining the
learning curve yeilds an insightful result.

\showlearning{Tree_c_2_l_5}

It is clear that the number of groups should have been chosen to be 5,
however, due to the non-greedy nature of the VB algorithm, it errored
on the side of a larger number of simpler groups. To patch this issue,
we introduce a modification to VB to make it greedy. Using this new greedy
approach, we can see the results of re computing the groups for $k=5$

\begin{minipage}{\linewidth}
\showbayes{Tree_c_2_l_5_greedy}
\vspace{10pt}
\end{minipage}

The greedy approach allows VB to value fewer groups as being simpler than
having many more groups. Using this new, greedy approach, we may re-model
the results to rebuild the plot from above.

\includesvg[width=\linewidth]{kvslev_greedy}

The exponential relationship is easier to see now that the graph
has been fixed.


\break
\section*{Ring Graphs}
Ring graphs are graphs which consist of cliques of nodes
around a ring. These graphs are trivial to find groups in
just by looking at them.


\subsection*{Poisson Distrubted Random Graph}

\subsection*{Configuration Model of Existing Graphs}

\subsubsection*{Karate Club}

\subsubsection*{Sarengeti Food Web}

\subsubsection*{Political Books}

\end{document}

Finally, no discussion would be complete without a discussion of
the applications and interpretations of these results on real networks.

For example, one of the most interesting networks seen for both algorithms
is the Serengeti Food Web\cite{Baskerville2011}.

\showgraphwithtitle{Serengeti Food Web}{serengeti-foodweb}

In this network, MDL found 4 solid groups. It seems to have detected
two ``metagroups'', one for each side. Each metagroups is then broken up
into a core-periphery structure with two subgroups. It seems that MDL
has found two ecosystems with their own set of predators and prey.

In this example, VB is not so lucky. It has not detected any structure. This
is very apparent from the complexity curve.

\showlearning{serengeti-foodweb}

\showgraphwithtitle{Zachary Karate Club}{karate}

The famous karate club network\cite{Zachary1977} is almost the opposite of the Serengeti
Foodweb. In this network, MDL does not find enough structure to merit the
creation of more than one group. However, VB found enough structure to
find five groups. Like MDL in the food web, VB has found 3 top level groups
of which two are broken into a core-periphery structure.

The dolphin data set\cite{Lusseau2003}

This graph is an interesting one because MDL has managed to find the 'correct'
partition. This is great, but in fact VB found a more interesting set of
things. The MDL algorithm seems to have found, like in the previous, certain
``metagroups'' that are sub-groups of the groups found by the MDL algorithm.

\showgraphwithtitle{Dolphins}{dolphins}

Without more knowledge of data within the network. It is hard for us to tell
what VB actually found. Perhaps it found the same two groups that MDL found
along with data about the social classes of the dolphins within those two
groups. Whatever the reason, Variational Bayes finds many more groups than MDL,
while retaining the ability to group these groups to create the same groups
that MDL found.

\section*{Conclusions}

There is no clear winner between MDL and VB. Both algorithms do a reasonable
job on some networks, while not so reasonable on others. The algorithm of
choice is dependent on the desired behavior. We have found that in general MDL
behaves in a more predictable way than VB. This may be exemplified in many of
the previous examples by the variation seen in the VB results. For example, for
some graphs with little to no structure will either cause VB to diverge into
assigning as many groups as there are vertices, or only one group is detected.
MDL on the other hands will always discover just one group for graphs with
inherently little structure.

VB excels at finding many groups, and when $k$ converges, many high resolution
structures may be picked out, generating much more information than MDL can. In
the Karate Club and Dolphins networks while MDL only picked out two or three
groups, VB was able to, in addition to finding those groups, recurse into those
groups and find subgroups, making it much more flexible as a whole.

With the flexibility VB offers, there a very real possibility of over fitting
the data, which is what was seen over and over again in the previous networks.
With this, MDL is much better at picking out the macro structure of the network
reducing the sway of random noise.

Over all, neither should always be used above the other. Rather, the modeler
should compare the pros and cons of each algorithm and pick the best for the
model and the data being obtained from the graph and what the potential risks
are



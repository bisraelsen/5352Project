\documentclass[twocolumn,twoside]{IEEEtran}

\usepackage{graphicx}
\usepackage{svg}

\newread\myinput

\newcommand{\showbayes}[1]{
% We use '\jobname.temp' to create a uniquely-named temporary file
\immediate\write18{cat ../Bayes/#1/net.txt.k > '\jobname.temp'}
\openin\myinput=\jobname.temp
% The group localizes the change to \endlinechar
\bgroup
  \endlinechar=-1
  \read\myinput to \localline
  % Since everything in the group is local, we have to explicitly make the
  % assignment global
  \global\let\kbayes\localline
\egroup
\closein\myinput
% Clean up after ourselves
\immediate\write18{rm -f -- '\jobname.temp'}

\newread\myinput

\center
Bayes
\includegraphics[width=\linewidth]{../Bayes/#1/#1.png}
$K=\kbayes$

}

\newcommand{\showmdl}[1] {
% We use '\jobname.temp' to create a uniquely-named temporary file
\immediate\write18{../get.py ../MDL/#1/mat.p > '\jobname.temp'}
\openin\myinput=\jobname.temp
% The group localizes the change to \endlinechar
\bgroup
  \endlinechar=-1
  \read\myinput to \localline
  % Since everything in the group is local, we have to explicitly make the
  % assignment global
  \global\let\kmdl\localline
\egroup
\closein\myinput
% Clean up after ourselves
\immediate\write18{rm -f -- '\jobname.temp'}

\center
MDL
\includegraphics[width=\linewidth]{../MDL/#1/#1.png}
$K=\kmdl$

}

\newcommand{\showgraph}[1]{
\vspace{10pt}
\begin{minipage}{0.45\linewidth}

\showmdl{#1}

\end{minipage}
\begin{minipage}{0.45\linewidth}

\showbayes{#1}

\end{minipage}
\vspace{10pt}

\def\currentgraph{#1}
}

\newcommand{\showlearning}[1]{
    \includegraphics[width=\linewidth]{../Bayes/#1/figure.png}
}

% \usepackage{minipage}

\nonstopmode \title{Comparison of Variational Bayes and Minimum Description
Length Algorithms for Determining $k$ in Undirected Networks}

\author{Brett Israelsen & Joshua Rahm}


\begin{document}
\maketitle

\begin{abstract}
We have compared two different algorithms for discovering
the number of groups, $k$, in a network given no prior information about the
structure of the network. The two algorithms we have tested in this project are
the Minimum Description Length (MDL) and Variational Bayes (VB).
\end{abstract}

\section*{Introduction}

Sometimes, when given a network, it is useful to find community structure in a
network in order to be able to do further analysis into the structure of the
network. Many times, little to no prior information is given about the state of
the network is known. Given this as the case, standard modularity measures like
greedy agglomeration and the KL heuristic are more difficult to use because of their
sensitivity to degree such as in the famous Karate Club network.

To combat this, there are several algorithms used to estimate the number of groups ($k$)
in a network, an from there, a reduction algorithm may be used to determine the association
of each vertex in each of the $k$ groups.

In this paper, we have compared the results obtained from two such algorithms;
Variational Bayes and Minimum Description Length.

\section*{Background}


\section*{Approach}

\section*{Results \& Discussion}



% For tree graphs:
%
% n_i over N for each group i as a histogram
% Distribution of level by community
% 
% For Ring Graphs:
%
%   n_i over n_c
%   m_i over (choose n_c 2)
%   average degree for MDL
%
% Price:
%   
% Poisson:
%   Average degree of each group
\subsection*{\underline{Well-Known Graphs}}

\newcommand{\inputsamepage}[1] {
    \begin{minipage}{\linewidth}
        \input{#1}
    \end{minipage}
}

We have run these algorithms on graphs which are well known. And have analyzed the
results.

\subsection*{Adjective-Nouns}
\inputsamepage{include/adjnoun.tex}

\subsection*{Bow-tie}
\inputsamepage{include/bowtie.tex}

\subsection*{Dolphins}
\inputsamepage{include/dolphins.tex}

\subsection*{Les Mis\'{e}rables}
\inputsamepage{include/lesmis.tex}

\subsection*{Football}
\inputsamepage{include/football.tex}

\subsection*{Zachary Karate Club}
\inputsamepage{include/karate.tex}

\subsection*{Political Books}
\inputsamepage{include/polbooks.tex}

\subsection*{Serengeti Foodweb}
\inputsamepage{include/serengeti-foodweb.tex}

\break
\section*{Binary Trees}
\begin{minipage}{0.45\linewidth}
MDL
\centering
\includegraphics[width=\linewidth]{../MDL/Tree_c_6_l_6/Tree_c_6_l_6.png}
$K=18$

\end{minipage}
\begin{minipage}{0.45\linewidth}
Bayes
\centering
\includegraphics[width=\linewidth]{../Bayes/Tree_c_6_l_6/Tree_c_6_l_6.png}
$K=39$
\end{minipage}
\vspace{10pt}

Once again, VB finds far more groups than the DL algorithm. However, the VB
did not max out the number, however, it came close to being maxing out and
it could be due to simple noise that the algorithm chose 39 over 40.


\break
\section*{Ring Graphs}
Ring graphs are graphs which consist of cliques of nodes around a ring. To
construct the ring graphs, there are two parameters we have defined: the clique
size $c$ and the number of cliques in the ring $g$

Some examples of such graphs are as follows.

\begin{minipage}{\linewidth}

\showgraphwithparams{
{\begin{align*}
c &: 4 \\
g &: 2
\end{align*}}
}{RLT_c_4_g_2}

\showgraphwithparams{
{\begin{align*}
c &: 4 \\
g &: 10
\end{align*}}
}{RLT_c_4_g_10}

\showgraphwithparams{
{\begin{align*}
c &: 40 \\
g &: 30
\end{align*}}
}{RLT_c_40_g_30}

\end{minipage}

Sometimes, with these graphs, Bayes and MDL under estimate the number of
groups, however, often, Bayes does a good job detecting an expected group
structure. In order to further understand how each algorithm detects
these groups and when they find an unexpected answer, we calculated the
following plot which shows the number of groups detected as a function of
the number of groups around the ring, with the clique size fixed to
$c=4$.

\begin{minipage}{\linewidth}
\includesvg[width=\linewidth]{k_vs_g_c_4}
\end{minipage}

MDL never does a great job discovering the number of groups, so for these
graphs, Bayes is generally going to be the better choice; however, this
is not without reservation as there is some critical point at $g=24$ that
causes VB to stop finding more groups. The mechanism for this is not fully
understood.

Increasing the number of nodes per clique changes significantly the
results.

\begin{minipage}{\linewidth}
\includesvg[width=\linewidth]{k_vs_g_c_10}
\end{minipage}

Here it is obvious that MDL does better, keeping up with VB for $g < 20$,
but falls far behind beginning at the critical point of $g=20$.



\subsection*{Poisson Distrubted Random Graph}

\subsection*{Configuration Model of Existing Graphs}

\subsubsection*{Karate Club}

\subsubsection*{Sarengeti Food Web}

\subsubsection*{Political Books}

\end{document}
